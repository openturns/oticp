% 
% Permission is granted to copy, distribute and/or modify this document
% under the terms of the GNU Free Documentation License, Version 1.2
% or any later version published by the Free Software Foundation;
% with no Invariant Sections, no Front-Cover Texts, and no Back-Cover
% Texts.  A copy of the license is included in the section entitled "GNU
% Free Documentation License".

%%%%%%%%%%%%%%%%%%%%%%%%%%%%%%%%%%%%%%%%%%%%%%%%%%%%%%%%%%%%%%%%%%%%%%%%%%%%%%%%%%%%%%%%%% 
\section{User Manual}

This document gives details on the new methods proposed by the python scripts. 



\subsection{Scripts required}

We describe here the python scripts required by the module : 
\begin{itemize}
  \item $IntegralUserDefined.py$, which implements the IntegralUserDefined class. This class is the UserDefined already contained in Open TURNS, specialised for integral variables.
  \item $IntegralUserDefinedFactory.py$, which gives the functionnalities to adjust a IntegralUserDefined variable to a set of data.
  \item $IntegralCompoundPoisson.py$ which implements the integral compound Poisson distribution.
  \item $Polynomial.py$ which gives functionnalities to manipulate sparse polynomials.
  \item $t\_IntegralUserDefined\_std.py$ which is the test file associated to the IntegralUserDefined class.
  \item $t\_IntegralUserDefinedFactory\_std.py$ which is the test file associated to the IntegralUserDefinedFactory class.
  \item $t\_IntegralCompoundPoisson\_validation.py$ which is the test file associated to the OTICP class.
\end{itemize}


\subsection{IntegralUserDefined}

\begin{description}

\item[Usage :] $IntegralUserDefined(range, weight)$

\item[Arguments :]  \strut
  \begin{description}
  \item $range$ : python list of integers or integer collection of type $UnsignedIntegerCollection$. For example, $range = [5,7,32]$ or $range = UnsignedIntegerCollection([5,7,32])$
  \item $weight$ :  python list of the associated weights or a $NumericalPoint$. If the weights are not normalised, they are authomatically normalised by Open TURNS. For example, $weight = [0.1, 0.2, 0.3, 0.4]$ or  $weight = NumericalPoint([0.1, 0.2, 0.3, 0.4])$.
  \end{description}

\item[Value :] a $IntegralUserDefined$, which is a variable which range ($range$) is  finite with integer values, associated to the weights given in $weight$.

\item[Some methods :]  \strut
  \begin{description}

  \item $getRange$
    \begin{description}
    \item[Usage :] $getRange()$
    \item[Arguments :] none
    \item[Value :]  a $UnsignedIntegerCollection$, which is a python list of integers which are the positive integer values of the range.
    \end{description}
    \bigskip


  \item $getWeights$
    \begin{description}
    \item[Usage :] $getWeights()$
    \item[Arguments :] none
    \item[Value :]  a $NumericalPoint$, the python list of the associated weights. 
    \end{description}
    \bigskip



  \item $getNormalizedWeights$
    \begin{description}
    \item[Usage :] $getNormalizedWeights()$
    \item[Arguments :] none
    \item[Value :]  a $NumericalPoint$, the python list of the associated normalised weights.
    \end{description}

    \end{description}
 
\end{description}





%%%%%%%%%%%%%%%%%%%%%%%%%%%%%%%%%%%%%%%%%%
\subsection{IntegralUserDefinedFactory}


\begin{description}

\item[Usage :] $IntegralUserDefinedFactory$ is a static class which is  used through its unique method $buildImplementation$.

\item[One method :]  $buildImplementation$
    \begin{description}

    \item[Usage :] $buildImplementation(echantillon)$

    \item[Arguments :]  $sample$ : un $NumericalSample$ which represents a numerical sample of the random variable. Its values must be integers.

    \item[Value :]  a $IntegralUserDefined$, a random variable which range is defined by the values f the $sample$. The probability of $k$ is estimated by  $ p_k = \displaystyle \frac{N_k}{n}$ where  $n$  is the size of $sample$ and $N_k$ the number of values of the $sample$ equal to $k$.

    \end{description}
 

 
\end{description}



%%%%%%%%%%%%%%%%%%%%%%%%%%%%%%%%%%%%%%%%%%
\subsection{IntegralCompoundPoisson}



\begin{description}

\item[Usage :] strut
  \begin{description}
    \item $IntegralCompoundPoisson(atomDistribution, theta)$
    \item $IntegralCompoundPoisson(atomDistribution, theta, param)$
  \end{description}

\item[Arguments :]  \strut
  \begin{description}
    \item $atomDistribution$ : a $IntegralUserDefined$ distribution
    \item $theta	$ : a strictly positive real
    \item $param$ : an integer. By default, $param = 10$. This parameter is such that $m = 2^{param}$ where $m$ is defined in the relation (\ref{pnApprox2}) with  the relation $m = 2^{param}$.
  \end{description}

\item[Value :] a $IntegralCompoundPoisson$, which the variable $Y$ defined as : 

\begin{equation}
Y = \left( \displaystyle \sum_{i=1}^{N}X_i \right)\fcar{N\geq1}
\end{equation}
where $N$ is a poisson distributed variable, the variables $X_i$  are discrete, with finite range and integer values, identically distributed, mutually independent and also independent of $N$.\\

When a variable $Y$ of type $IntegralCompoundPoisson$ is created, all the values $p_k = \mathcal{P}(Y=k)$ are authomatically evaluated for $k \in [0, m-1]$ with $m = 2^{param}$. If the User asks for the evaluation of $p_n$ for $n \geq m$, then Open TURNS authomatically re-evaluates  all the values   $p_k$ for $k \in [0, N]$  with $N   = 2^{E[log_2 n]+1}$ (ie the first power of 2 which is strictly  $>n$).

\item[Some methods :]  \strut
  \begin{description}

  \item $getAtomDistribution$
    \begin{description}
    \item[Usage :] $getAtomDistribution()$
    \item[Arguments :] none.
    \item[Value :]  the finite discrete distribution of the variables $X_i$ of type $IntegralUserDefined$
    \end{description}
    \bigskip


  \item $getTheta$
    \begin{description}
    \item[Usage :] $getTheta()$
    \item[Arguments :] none.
    \item[Value :]  un positive real, the Poisson distribution parameter.
    \end{description}
    \bigskip

  \item $getLog2Cache$
    \begin{description}
    \item[Usage :] $getLog2Cache()$
    \item[Arguments :] none.
    \item[Value :]  an integer, the parameter $param$. By default, $param = 10$. This parameter is such that $m = 2^{param}$ where $m$ is defined in the relation (\ref{pnApprox2}) with  the relation $m = 2^{param}$.
    \end{description}
    \bigskip


  \item $getM$
    \begin{description}
    \item[Usage :] $getM()$
    \item[Arguments :] none.
    \item[Value :]  the integer $m$ such that  $m = 2^{param}$. By default, $m=2^{10}$. $m$ is defined in the relation    (\ref{pnApprox2}).
    \end{description}
    \bigskip

  \item $getRealization$
    \begin{description}
    \item[Usage :] $getRealization()$
    \item[Arguments :] none.
    \item[Value :]  an integer inside the of $Y$.
    \end{description}
    \bigskip


  \item $getNumericalSample$
    \begin{description}
    \item[Usage :] $getNumericalSample(size)$
    \item[Arguments :] $size$ : an integer, the number of realistaions to generate a numerical sample of $Y$.
    \end{description}
    \bigskip



  \item $getMean$
    \begin{description}
    \item[Usage :] $getMean()$
    \item[Arguments :]  none.
    \item[Value :]  the mean of the random variable.
    \end{description}
    \bigskip


  \item $getCovariance$
    \begin{description}
    \item[Usage :] $getCovariance()$
    \item[Arguments :]  none.
    \item[Value :]  the variance of the random variable.
    \end{description}
    \bigskip


  \item $getStandardDeviation$
    \begin{description}
    \item[Usage :] $getStandardDeviation()$
    \item[Arguments :]  none.
    \item[Value :]  the standard deviation of the random variable.
    \end{description}
    \bigskip
 

  \item $computeCDF$
    \begin{description}
    \item[Usage :] \strut
  \begin{description}
      \item $computeCDF(value)$
      \item $computeCDF(value, True)$
  \end{description}
    \item[Arguments :] \begin{description}
      \item $value$ : an integer.
      \item $True$ : a boolean wich is False when not mentioned.
  \end{description}
    \item[Value :]  a real between 0 and 1. When the boolean is FALSE, the function computes the CDF at $value$. In the other case, it computes the tail of the CDF at $value$.
    \end{description}
    \bigskip


  \item $computePDF$
    \begin{description}
    \item[Usage :] $computePDF(value)$
    \item[Arguments :]  $value$ : an integer.
    \item[Value :]  a real which is the PDF value at point $value$.
    \end{description}
    \bigskip

  \item $computeQuantile$
    \begin{description}
    \item[Usage :] $computeQuantile(q)$
    \item[Arguments :]  $q$ : a real between 0 and 1. 
    \item[Value :]  the quantile of order $q$.
    \end{description}
    \bigskip


  \item $drawPDF$
    \begin{description}
    \item[Usage :] $drawPDF(xMin, xMax)$
    \item[Arguments :] $(xMin, xMax)$ : the min and max values of the graph.
    \item[Value :]  a $Graph$ which contains the PDF curve.
    \end{description}
    \bigskip


  \item $drawCDF$
    \begin{description}
    \item[Usage :] $drawCDF(xMin, xMax)$
    \item[Arguments :] $(xMin, xMax)$ : the min and max values of the graph.
    \item[Value :]  a $Graph$ which contains the CDF curve.
    \end{description}

    \end{description}
 
\end{description}



%%%%%%%%%%%%%%%%%%%%%%%%%%%%%%%%%%%%%%%%%%
\subsection{Polynomials}

The file $Polynomials.py$ contains a set of functions linked to the manipulation of polynomials.



\begin{description}


  \item $denseToSparse$
    \begin{description}
    \item[Usage :] $denseToSparse(polynomial)$
    \item[Arguments :] $polynomial$ : a $UniVariatePolynomial$ which is the polynomials class in Open TURNS.
    \item[Value :]  $degrees, coefficients$ : respectively a $UnsignedIntegerCollection$ and $NumericalPoint$ which are the degrees of the monoms which coefficient is not nul and the associated coefficient.
    \end{description}
    \bigskip


  \item $buildUniVariatePolynomial$
    \begin{description}
    \item[Usage :] $buildUniVariatePolynomial(degrees, coefficients)$
    \item[Arguments :] $degrees, coefficients$ : respectively a $UnsignedIntegerCollection$ and $NumericalPoint$ which gives the degrees of the monoms which coefficient is not nul and the associated coefficient.
    \item[Value :]  $polynomial$ : a $UniVariatePolynomial$ which is the polynomials class in Open TURNS, with a full representation : the coefficients of all the monoms are stocked, some eventually equal to 0.
    \end{description}
    \bigskip


  \item $truncateUniVariatePolynomial$
    \begin{description}
    \item[Usage :] $truncateUniVariatePolynomial(polynomial, truncation)$
    \item[Arguments :]  \strut
    \begin{description}
      \item $polynomial$ : a $UniVariatePolynomial$ which is the polynomials class in Open TURNS.
      \item $truncation$ : an integer.
    \end{description}
    \item[Value :]  a $UniVariatePolynomial$ : this is the initial polynomials truncated to the degree $truncation+1$. The degree of the truncated polynomials is $truncation$.
    \end{description}

 
\end{description}


